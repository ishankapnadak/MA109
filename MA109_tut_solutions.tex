\documentclass[12pt]{article}
\usepackage{amsmath, amssymb, amsfonts, amsthm, mathtools,mathrsfs}
\usepackage{thmtools}
\usepackage[utf8]{inputenc}
\usepackage[inline]{enumitem}
\usepackage[colorlinks=true]{hyperref}
\usepackage{tikz}
\usetikzlibrary{decorations.markings}
\usetikzlibrary{arrows.meta}
\usepackage{witharrows}
\usepackage[useregional, showdow]{datetime2}
\usepackage{physics}
\DTMlangsetup[en-GB]{abbr}
\usepackage{xcolor}


\setlength\parindent{0pt}
\usepackage{parskip}

\usepackage[framemethod=tikz]{mdframed}
\mdfdefinestyle{theoremstyle}{%
	% linecolor=gray,linewidth=1pt,%
	% frametitlerule=true,%
	frametitlebackgroundcolor=white,
	% backgroundcolor=  gray!20,	
	bottomline=false, topline=false, rightline=false, leftline=true,
	innerlinewidth=0.7pt, outerlinewidth=0.7pt, middlelinewidth=2pt, middlelinecolor=white, %
	innerleftmargin=6pt,
	% innertopmargin=-1pt,
	skipabove=10pt,
	% fontcolor=blue,
	% innerbottommargin=-0.5pt,
}
\mdtheorem[style=theoremstyle]{defn}[thm]{Definition}
\mdtheorem[style=theoremstyle]{thm}{Theorem}

\newcommand*{\doublerule}{\hrule width \hsize height 1pt \kern 0.5mm \hrule width \hsize height 2pt}
\newcommand{\doublerulefill}{\leavevmode\leaders\vbox{\hrule width .1pt\kern1pt\hrule}\hfill\kern0pt}
\def\ddfrac#1#2{\displaystyle\frac{\displaystyle #1}{\displaystyle #2}}


%\newcommand{\Res}{\operatorname{Res}}

\theoremstyle{definition}
% \newtheorem{thm}{Theorem}
% \numberwithin{thm}{section}
% \newtheorem{lem}[thm]{Lemma}
% \newtheorem{defn}[thm]{Definition}
% \newtheorem{prop}[thm]{Proposition}
% \newtheorem{cor}[thm]{Corollary}
% \newtheorem{ex}{Example}


\let\emptyset\varnothing

\usepackage{titlesec}
\titleformat{\section}[block]{\Large\filcenter\bfseries}{\S\thesection.}{0.25cm}{\Large}
\titleformat{\subsection}[block]{\large\bfseries\sffamily}{\S\S\thesubsection.}{0.2cm}{\large}

\usepackage[a4paper]{geometry}
\usepackage{lipsum}

\usepackage{cleveref}
\crefname{thm}{Theorem}{Theorems}
\crefname{lem}{Lemma}{Lemmas}
\crefname{defn}{Definition}{Definitions}
\crefname{prop}{Proposition}{Propositions}
\crefname{cor}{Corollary}{Corollaries}
\crefname{equation}{}{}

\usepackage{mdframed}
\newenvironment{blockquote}
{\begin{mdframed}[skipabove=0pt, skipbelow=0pt, innertopmargin=4pt, innerbottommargin=4pt, bottomline=false,topline=false,rightline=false, linewidth=2pt]}
{\end{mdframed}}
\newenvironment{soln}{\begin{proof}[Solution]}{\end{proof}}

\title{MA 109: Calculus - I\\\large{Tutorial Solutions}}
\author{Ishan Kapnadak}
\date{Autumn Semester 2020-21\\~\\Updated on: \textcolor{blue}{\DTMToday}}

\begin{document}
\tikzset{lab dis/.store in=\LabDis,
  lab dis=-0.4,
  ->-/.style args={at #1 with label #2}{decoration={
    markings,
    mark=at position #1 with {\arrow{>}; \node at (0,\LabDis) {#2};}},postaction={decorate}},
  -<-/.style args={at #1 with label #2}{decoration={
    markings,
    mark=at position #1 with {\arrow{<}; \node at (0,\LabDis)
    {#2};}},postaction={decorate}},
  -*-/.style args={at #1 with label #2}{decoration={
    markings,
    mark=at position #1 with {{\fill (0,0) circle (1.5pt);} \node at (0,\LabDis)
    {#2};}},postaction={decorate}},
  }
\maketitle
\tableofcontents
\newpage\section{Week 1}
\begin{center}
	25th November, 2020
\end{center}
\textbf{Sheet 1.}
\begin{enumerate}[leftmargin=*]
    \itemsep0.5em
    \item[2 (iv)] $\displaystyle\lim_{n\to \infty}(n)^{1/n}.$
    \begin{soln}
        We will utilise the fact that $n^{1/n} \geq 1$ for all $n \in \mathbb{N}$. (Why is this true?) We define $h_n \coloneqq n^{1/n} - 1$. Then, $h_n \geq 0$ for all $n \in \mathbb{N}$. For $n\geq2$, we have
        \[
            n = (1+h_n)^n \geq 1 + \binom{n}{1} h_n + \binom{n}{2} h_n^2 > \binom{n}{2} h_n^2 = \frac{n(n-1)}{2} h_n^2
        \]
        Cancelling out the $n$'s, we get
        \[
            h_n^2 < \frac{2}{n-1} \implies h_n < \sqrt{\frac{2}{n-1}}
        \]  
        Thus for $n \geq 2$, we have
        \[
            0 \leq h_n < \sqrt{\frac{2}{n-1}}
        \]
        Notice that the limit of the sequence on the right exists and is equal to $0$. Thus, utilising Sandwich Theorem, we get that $\lim\limits_{n \to \infty} h_n = 0$. Recalling how we defined $h_n$, we get $\lim\limits_{n \to \infty} n^{1/n} = 1$.
    \end{soln}
    
    \newpage
    
    \item[3 (ii)] Prove that the sequence $a_n \coloneqq \left\{ (-1)^n \left( \ddfrac{1}{2} - \ddfrac{1}{n} \right) \right\}_{n \geq 1}$ is not convergent.
    
    \begin{soln}
        We will prove this result by contradiction. First, observe that the sequence $b_n \coloneqq \ddfrac{(-1)^n}{n}$ is convergent and its limit is $0$. This is true because its absolute value behaves the same way as $\ddfrac{1}{n}$ (try proving this with the $\epsilon-N$ definition to work out the details). We also know that the sequence $\{(-1)^n\}_{n \geq 1}$ is not convergent. (Why?) Now, let us assume that the given sequence $(a_n)$ converges. We have
        \[
            a_n \coloneqq \left\{ (-1)^n \left( \ddfrac{1}{2} - \ddfrac{1}{n} \right) \right\} = \frac{(-1)^n}{2} - \frac{(-1)^n}{n}
        \]
        We also know that the the sum of two convergent sequences is convergent. Since $a_n$ is assumed to be convergent and $b_n$ is convergent, we have that $c_n \coloneqq a_n + b_n = \ddfrac{(-1)^n}{2}$ must also converge. However, the convergence of $c_n$ implies that the sequence $(-1)^n$ also converges. Hence, we arrive at a contradiction and thus, the sequence $(a_n)$ is not convergent.
        
    \end{soln}
    
    \newpage
    
    \item[5 (iii)] Prove that the following sequence is convergent by showing that it is monotone and bounded. Also find its limit. 
    \[
        a_1 = 2, a_{n+1} = 3 + \frac{a_n}{2} \; \forall n \in \mathbb{N}
    \]
    
    \begin{soln}
        We first claim that $a_n < 6$ for all $n \in \mathbb{N}$. To prove this, we will use mathematical induction. The base case, $n=1$ is immediate as $2<6$. Assume that the claim holds for some $n=k$. Now, 
        \[
            a_{k+1} = 3 + \frac{a_k}{2} < 3 + \frac{6}{2} = 6
        \]
        By induction, the claim follows. Hence, $a_n$ is bounded above. 
        \medskip
        
        Next, we claim that $a_{n+1} > a_n$ for all $n \in \mathbb{N}$. We have
        \[
            a_{n+1} - a_n = 3 - \frac{a_n}{2} = \frac{6-a_n}{2}
        \]
        We just showed that $a_n < 6$ for all $n \in \mathbb{N}$. It thus follows that $a_{n+1} > a_n$ for all $n \in \mathbb{N}$. Hence, $(a_n)$ is a monotonically increasing sequence that is bounded above. Thus, it must converge. To find the limit of $(a_n)$, we utilise the fact that $\lim\limits_{n \to \infty} a_{n+1} = \lim\limits_{n \to \infty} a_n$ (Sheet $1$ : Problem $6$). Let $L$ denote the limit of $(a_n)$. Taking the limit of the recursive definition (and using some limit properties), we have that
        \[
            L = 3 + \frac{L}{2} \implies L = 6
        \]
        Thus, the sequence $(a_n)$ converges to 6. (Notice that this was the upper bound we chose for $(a_n)$)
    \end{soln}
    
    \newpage
    
    \item[7] If $\lim\limits_{n \to \infty} a_n = L \neq 0$, show that there exists $n_0 \in \mathbb{N}$ such that
    \[
        \abs{a_n} \geq \frac{\abs{L}}{2}, \quad \forall n \geq n_0
    \]
    
    \begin{soln}
        We will use the $\epsilon-N$ definition to prove this result. Choose $\epsilon = \ddfrac{\abs{L}}{2}$. Since $L \neq 0$, we have $\epsilon > 0$. Now, as $a_n \to L$, there exists $n_0 \in \mathbb{N}$ such that $\abs{a_n - L} < \epsilon$ for all $n \geq n_0$. From triangle inequality, we have
        \[
            \abs{\abs{a_n} - \abs{L}} \leq \abs{a_n - L} < \epsilon \implies -\epsilon < \abs{a_n} - \abs{L} \quad \forall n \geq n_0
        \]
        Substituting the value of $\epsilon$, we get that
        \[
            \abs{a_n} > \frac{\abs{L}}{2}
        \]
        for all $n \geq n_0$, as desired.
    \end{soln}
    
    \item[9] For given sequences $\{a_n\}_{n \geq 1}$ and $\{b_n\}_{n \geq 1}$, prove or disprove the following statements:
    \begin{enumerate}[label = (\roman*)]
        \item $\{a_n b_n\}_{n \geq 1}$ is convergent if $\{a_n\}_{n \geq 1}$ is convergent.
        \item $\{a_n b_n \}_{n \geq 1}$ is convergent if $\{a_n\}_{n \geq 1}$ is convergent and $\{b_n\}_{n \geq 1}$ is bounded.
    \end{enumerate}
    
    \begin{soln}
        This is a relatively short question. Both the statements are \textbf{false}. Verify that $a_n \coloneqq 1$ and $b_n \coloneqq (-1)^n$ acts as a counterexample for both the statements.
    \end{soln}
    
    \newpage
    
    \item[11] Let $f,g \colon (a,b) \rightarrow \mathbb{R}$ be functions and suppose that $\lim\limits_{x \to c} f(x) = 0$ for $c \in [a,b]$. Prove or disprove the following statements. 
    \begin{enumerate}[label = (\roman*)]
        \item $\lim\limits_{x \to c} \left[ f(x) g(x) \right] = 0$.
        \item $\lim\limits_{x \to c} \left[ f(x) g(x) \right] = 0$ if $g$ is bounded.
        \item $\lim\limits_{x \to c} \left[ f(x) g(x) \right] = 0$ if $\lim\limits_{x \to c} g(x)$ exists.
    \end{enumerate}
    \begin{soln}
    
        \begin{enumerate}[label = (\roman*), labelwidth=!, labelindent=0pt]
            \item This statement is \textbf{false}. As a counterexample, define $a=-1, b=1$ and $c=0$. Define $f,g \colon (-1,1) \rightarrow \mathbb{R}$ as
            \[
                f(x) = x \quad \text{ and } \quad g(x) = \begin{cases}
                    1 & \text{ if } x=0 \\
                    \frac{1}{x^2} & \text{ if } x \neq 0
                \end{cases}
            \]
            Clearly, $\lim\limits_{x \to 0} f(x) = 0$. However, $\lim\limits_{x \to 0} \left[ f(x) g(x) \right]$ does not exist.
            
            \item This statement is \textbf{true}. Since $g$ is bounded, there exists $M > 0$ such that
            \[
                \abs{g(x)} \leq M 
            \]
            for all $x \in (a,b)$. Thus, we have 
            \[
                0 \leq \abs{f(x) g(x)} \leq M \abs{f(x)}
            \]
            for all $x \in (a,b)$. Using Sandwich Theorem, we see that
            \[
                \lim_{x \to c} \; \abs{f(x) g(x)} = 0
            \]
            which in turn implies that
            \[
                \lim_{x \to c} \; [f(x) g(x)] = 0
            \]
            
            \item This statement is \textbf{true}. Since $\lim\limits_{x \to c} g(x)$ exists, we have $\lim\limits_{x \to c} \left[ f(x) g(x) \right] = \lim\limits_{x \to c} f(x) \cdot \lim\limits_{x \to c} g(x) = 0$.
        \end{enumerate}
    \end{soln}
    
    %\item[13 (iii)] Discuss the continuity of the following function :
    %\[
        %f(x) = \begin{cases}
            %\frac{x}{\lfloor x \rfloor} & \text{ if } 1 \leq x < 2 \\
            %1 & \text{ if } x = 2 \\
            %\sqrt{6-x} & \text{ if } 2 < x \leq 3
        %\end{cases}
    %\]
    
    %\begin{soln}
        %We only need to check continuity of $f$ at $x=2$. It is trivially at all other points in the interval $[1,3]$. Evaluating the left hand side limit, we see that $\lim\limits_{x \to 2^-} f(x) = \lim\limits_{x \to 2^-} \ddfrac{x}{1} = 2$. Evaluating the right hand side limit, we see that $\lim\limits_{x \to 2^+} f(x) = \lim\limits_{x \to 2^+} \sqrt{6-x} = 2$. Thus, the limits on both sides exist and are equal to $2$. Hence, $\lim\limits_{x \to 2}f(x) = 2$. Note however that $f(2) = 1 \neq 2 =  \lim\limits_{x \to 2}f(x)$. Thus, $f$ is \textbf{not} continuous at $x=2$ (it has a removable discontinuity). 
    %\end{soln}
\end{enumerate}

\end{document}
